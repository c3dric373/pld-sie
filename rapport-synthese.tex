%% Erläuterungen zu den Befehlen erfolgen unter
%% diesem Beispiel.

%% Article Template
\documentclass[]{scrartcl}


%% UTF8 Encoding
\usepackage[utf8]{inputenc}
\usepackage[T1]{fontenc}
%\usepackage[margin=5cm]{geometry}% http://ctan.org/pkg/geometry
\usepackage{lmodern}
\usepackage{subcaption}
\usepackage{setspace}
\usepackage[french]{babel}
\usepackage{csquotes}
\usepackage{pdfpages}
% appendix
\usepackage[titletoc]{appendix}

\setstretch{0.99}

%Tabellen mit fixen Breiten
\usepackage{tabularx}

%% Grafik
\usepackage{graphicx} 
\newcommand{\name}{Insta.edit}

%% Links im PDF
\usepackage{hyperref}
\usepackage{rotating}
\usepackage{lscape}
\usepackage{fancyhdr}
\usepackage{float}
\usepackage[titles]{tocloft}
\pagestyle{fancy}
\rhead{\includegraphics[scale=0.15]{img/insa-logo}}
\chead{\Large -SIE- }
\lhead{Hexanome 4222}

\title{Rapport Synthese\\
\subtitle{}
\author{PLD: SIE\\
Professeurs:\
\ \\
Version 1.0}
\date{16/10/2019}}


\begin{document}

\maketitle

\begin{figure}[h]
	\centering
  \includegraphics[width=0.5\textwidth]{img/insa-logo}
	\label{fig:logo}
\end{figure}


\begin{center}
  \begin{tabular}{ | l | r | }
    \hline
    \textbf{Corentin Laharotte}\\ \hline
    \textbf{Cédric Milinaire }\\ \hline
    \textbf{Felix Castillon}\\ \hline
    \textbf{Roxane Debord}\\ \hline
     \textbf{Grazia Giulia Ribbeni}\\ \hline
     \textbf{Ousmane Touat} \\ \hline
     \textbf{David Hamidovic} \\ \hline


  \end{tabular}
\end{center}

\thispagestyle{empty}
\pagebreak
\vspace*{10pt}
\tableofcontents
\listoffigures
\newpage

\section{Comprehension de l'entreprise}
\subsection{Contexte}
La société SPIE est fondée en 1900 sous le nom de Société parisienne pour l'industrie des chemins de fer et des tramways électriques. Elle devient la Société parisienne pour l'industrie électrique  (SPIE) en 1946. La société SPIE s'occupe de l'étude, la réalisation et la maintenance d'équipements dans différents secteurs : énergie, transport, réseaux extérieurs ,installations éléctriques, mécanique. La filliale Sud-Est , en particulier, a généré un chiffre d'affaire de 356 Millions d'euros en 2010.

Le domaine sur lequel s'applique le cahier des charges est l'activité de maintenance avec des types variés : cela va de la maintenance informatique jusqu'à la maintenance d'infrastucture.SPIE peut être amené à gérer des centaines de contrats de maintenance très divers notamment dans le cadre de la maintenance préventive. L'entreprise souhaite améliorer le processus de gestion de la maintenance.  


\section{Comprehension des Processus}
Nous nous intéressons ici au processus de gestion des contrats de maintenance et services avec ses sous-processus qui démarre lorsqu'une opportunité d'avenant ou d'un contrat de service arrive.\\

Le premier sous processus "Offre et revue d'offre" concerne l'analyse de l'offre du point de vue des ressources existantes et nécessaires, des risques et de la faisabilité. Cela peut entraîner l'arrêt du processus de maintenance entier si l'étude de faisabilité n'est pas satisfaisant. Si au contraire l'étude de faisabilité est à bilan positif une étude complète de la solution à proposer est effectuée et validée (ou pas) en interne, puis elle est transmise au client.\\

Le processus "Réalisation des travaux induits" démarre lorsque des travaux supplémentaires sont détectés au cours d'une maintenance. Après l'analyse des travaux induits et des aspects contractuel deux options sont possibles. Si le contrat inclut la réalisation des travaux induits, ceux-ci sont chiffrés et réalisé en chiffrage contrôlé. Sinon un nouveau devis est nécessaire. Dans les deux cas, la facturation a lieu une fois que toutes les validations ont été faites et les travaux induits ont été conclus.


\section{Propositions de solutions}
\end{document}
