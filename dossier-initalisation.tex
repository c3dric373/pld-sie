%% Erläuterungen zu den Befehlen erfolgen unter
%% diesem Beispiel.

%% Article Template
\documentclass[]{scrartcl}


%% UTF8 Encoding
\usepackage[utf8]{inputenc}
\usepackage[T1]{fontenc}
%\usepackage[margin=5cm]{geometry}% http://ctan.org/pkg/geometry
\usepackage{lmodern}
\usepackage{subcaption}
\usepackage{setspace}
\usepackage[french]{babel}
\usepackage{csquotes}
\usepackage{pdfpages}
% appendix
\usepackage[titletoc]{appendix}

\setstretch{0.99}

%Tabellen mit fixen Breiten
\usepackage{tabularx}

%% Grafik
\usepackage{graphicx} 
\newcommand{\name}{Insta.edit}

%% Links im PDF
\usepackage{hyperref}
\usepackage{rotating}
\usepackage{lscape}
\usepackage{fancyhdr}
\usepackage{float}
\usepackage[titles]{tocloft}
\pagestyle{fancy}
\rhead{\includegraphics[scale=0.15]{img/insa-logo}}
\chead{\Large -PLD SIE- }
\lhead{Hexanome 4222}

\title{SAP ERP
\subtitle{}
\author{PLD: SIE
Version 1.0}
\date{\today}}


\begin{document}

\maketitle

\begin{figure}[h]
	\centering
  \includegraphics[width=0.5\textwidth]{img/insa-logo}
	\label{fig:logo}
\end{figure}


\begin{center}
  \begin{tabular}{ | l | r | }
    \hline
    \textbf{Corentin Laharotte}\\ \hline
    \textbf{Cédric Milinaire }\\ \hline
    \textbf{Felix Castillon}\\ \hline
    \textbf{Ousmane Touat}\\ \hline
	\textbf{Grazia Giulia Ribbeni }\\ \hline
	\textbf{Roxane Debord} \\ \hline 
	    \textbf{David Hamidovic}\\ \hline
  \end{tabular}
\end{center}

\thispagestyle{empty}
\pagebreak
\vspace*{10pt}
\tableofcontents
\listoffigures
\newpage
\section{OBJET DU PROJET}
Demande de conception, développement et finalement le déploiement d'un système d'information (SI) chargé d'effectuer des opérations de surveillance de sites pétroliers dudit client. Ce SI permettra ainsi de signaler toutes les anomalies détectées et permettra également grâce à une application d'aide à la décision d'assister les responsables à effectuer les choix adéquats. L'ensemble de données collectées est stocké sur un serveur afin de pouvoir effectuer des traitements statistiques pour une meilleure planification des interventions et une meilleure surveillance. Ce système permettra également d'assurer tous les suivis des interventions effectuées par les sociétés de maintenance. Ces dernières pourront accéder à un ensemble de services de ce système central, afin de lister et localiser les opérations de maintenance à effectuer. \\

Il sera également possible d'effectuer des opérations de configuration et de maintenance a distance de l'ensemble des dispositifs matériel.\\

Ce SI sera composé de 3 lots:
\begin{itemize}
\item \textquote{Lot Applications}: développement d'application utilisateurs ( gestion, configuration des captuers, maintenance). Ldéveloppement se fait selon un cycle en V (SGG, SGD, Codage et test unitaires, test d'intégration, tests de niveau système). 
\item \textquote{Lot Matériel}: site central (serveurs et postes de travail), système des capteurs sur site
\item \textquote{Lot Procédure}: procédures d'exploitation formalisées
\end{itemize}
\ \\
L'équipe Milinaire Bodet prendra en charge le développement du lot Applications. La liste des livrables exacte est décrite dans le chapitre suivant. 
\newpage
\section{RESULTATS LIVRABLE ATTENDUS: PBS}

Etude préalable : 
\begin{itemize}
\item expression des besoins
\begin{itemize}
\item Dossier de synthèse
\item Benchmarking
\end{itemize}
\item cible fonctionnelle
\begin{itemize}
\item Matrice d'impacts
\item Architecture Applicative Cible v0
\item Matrice des interfaces v0
\end{itemize}
\item Atelier Avanade
\begin{itemize}
\item Matrice FIT/GAP v0
\end{itemize}
\item Construction de la solution
\begin{itemize}
\item Matrice FIT/GAP v1
\item Matrice des interfaces v1
\item Architecture Applicative Cible v1
\end{itemize}
\item Modélisation Détaillée des processus
\begin{itemize}
\item Rapport ARIS
\item Dossier de description de la solution
\end{itemize}
\item Contrôle Qualité
\begin{itemize}
\item Plan d'Assurance Qualité
\end{itemize}
\item [Plan projet]
\end{itemize}
\ \\
Gestion de Projet : 
\begin{itemize}
\item Dossier d'Initialisation
\item Tableau de bord d'avancement
\item Dossier Bilan
\item Ordre du jour
\end{itemize}

\section{OUTILS UTILISES}

\section{IDENTIFICATION DES ACTIVITES ET TACHES: WBS}
\subsection{Lises des Activites et des Taches}
\begin{center}
\begin{figure}[H]
\caption{Liste des taches}
\end{figure}
\end{center}
\subsection{Plan de charges}
\begin{center}
\begin{figure}[H]
\caption{Plan de charges}
\end{figure}
\end{center}
\subsection{Diagramme de Gant}
\begin{figure}[H]
\caption{Diagramme de Gantt}
\end{figure}
\section{ORGANISATION DE L'ÉQUIPE:OBS}
\begin{center}
	\begin{tabular}{| l | c | c |}
	\hline
	Role                     & Description & Personne affectée  \\ 	\hline
	Chef de Projet           &             & Cédric Milinaire                   	\\ \hline
	Assistant chef de Projet &             & Corentin Laharotte                  	\\ \hline
	Responsable Qualité      &             & Ousmane Touat                 	\\ \hline
	Responsable Aris         &             & Roxane Debord                 	\\ \hline
	Expert Métier            &             & Grazia Giula Ribbeni                 	\\ \hline
	Expert SAP               &             & David Hamidovic                 	\\ \hline
	Architecte               &             & Félix Castillon  \\ 		\hline
	\end{tabular}
\end{center}

\section{ANALYSE DES RISQUES}
\subsection{Perte de Bobby au court du projet}
Bobby étant le programmeur le plus expérimenté du projet, nous comptons sur lui pour une grande partie des taches. S’il lui arrive un contretemps et qu’il vient à nous quitter, nous devrions trouver des alternatives à cela. Nous ne pouvons faire appelle à d’autre programmeur pour palier à son manque. Les compétences en développement du chef de projet lui permettront de prendre une partie du travail laissé par Bobby. Il en résultera un retard certain mais moins important que prévu. Le retard serait de 10 à 40 jours en fonction de la date de départ.


\begin{appendices}
\end{appendices}
\end{document}
